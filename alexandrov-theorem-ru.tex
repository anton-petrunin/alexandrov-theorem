\documentclass[oneside,a4paper]{article}
\usepackage{ru-note}
\usepackage{indentfirst}
\hypersetup{
pdftitle={Теорема Александрова о вложении многогранников},
pdfauthor={Нина Лебедева и Антон Петрунин}
}


\begin{document}
%\pagestyle{empty}\renewcommand\includegraphics[2][{}]{}

\title{Теорема Александрова о\\ вложении многогранников}
\author{Нина Лебедева и Антон Петрунин}
\date{}
\maketitle

\begin{abstract}
Теорема Александрова даёт полное описание возможных внутренних метрик на поверхностях выпуклых многогранников.
Мы приводим набросок её доказательства.
\end{abstract}

\section{Введение}

Нас будет интересовать \emph{внутренняя метрика} (то есть метрика, задающаяся длинами кратчайших путей между точками) на поверхности выпуклого многогранника.

Напомним, что сумма углов при вершине многогранного выпуклого угла меньше $2\cdot \pi$; 
это утверждение можно найти в школьном учебнике \cite[§~325]{kiselyov}.

Нетрудно видеть, что поверхность выпуклого многогранника гомеоморфна сфере.
Из вышесказанного получаем, что поверхность выпуклого многогранника, наделённая естественной внутренней метрикой,
является примером \emph{многогранной метрики на двумерной сфере с суммой углов вокруг каждой вершины, не превосходящей $2\cdot\pi$}; 
метрика называется \emph{многогранной} если сфера допускает триангуляцию такую, что каждый её треугольник равен плоскому треугольнику.

Теорема Александрова гласит, что обратное верно, если включить в рассмотрение \emph{дважды покрытые многоугольники}.
Иначе говоря, мы допускаем многогранники, вырождающиеся в плоские многоугольники;
под их поверхностью понимается две копии многоугольника, склеенные по границе.

Везде далее мы предполагаем, что многогранник может вырождаться в плоский многоугольник.

\begin{thm}{Теорема Александрова}
\begin{enumerate}[I.]
\item\label{thm:exist} Многогранная метрика на сфере изометрична поверхности выпуклого многогранника тогда и только тогда, когда сумма углов при любой её вершине не превосходит $2\cdot\pi$.

\item\label{thm:unique}
Более того, с точностью до конгруэнтности выпуклый многогранник определяется внутренней метрикой на своей поверхности.
\end{enumerate}

\end{thm}

У Александра Даниловича много замечательных теорем, но с нашей точки зрения эта теорема особенно замечательна.
При этом её доказательство вполне элементарно, его можно объяснить любому, кто знаком с азами топологии.

Эта теорема имеет огромное число приложений.
В частности, она используется в доказательстве её обобщения \cite{alexandrov-1948},
дающего полное описание внутренних метрик на сфере, изометричных замкнутым выпуклым поверхностям  евклидова пространства (не обязательно многогранным).
Последняя теорема является фундаментальным результатом в разделе современной математики --- так называемой \emph{Александровской геометрии}.

Первая часть теоремы является основной, она называется \emph{теоремой существования}.
Вторая часть называется \emph{теоремой единственности}; это лишь небольшая вариация теоремы Коши о многогранниках.
(У Александрова имеется ещё и другая теорема единственности, обобщающая теорему Минковского о многогранниках.)

\begin{wrapfigure}{r}{30mm}
\vskip-4mm
\centering
\includegraphics{mppics/pic-10}
\vskip-0mm
\end{wrapfigure}

Согласно теореме выпуклый многогранник полностью определяется внутренней метрикой своей поверхности.
Значит, зная метрику на поверхности, можно в принципе узнать, где пройдут рёбра многогранника.
Однако на деле это сделать непросто.
Например, поверхность, склеенная из прямоугольника, как показано на картинке, определяет тетраэдр.
При этом некоторые линии склеек оказываются внутри граней тетраэдра и часть рёбер тетраэдра (штрихованные линии) вовсе не идут вдоль сторон прямоугольника.

Теорема доказана А.~Д. Александровым в 1941 году \cite{alexandrov-1941};
мы приведём набросок его доказательства. 
Полное доказательство лучше всего написано в книжке самого Александра Даниловича~\cite{alexandrov}.
Ещё одно доказательство, основанное на деформации трёхмерного многогранного пространства, было предложено Ю. А. Волковым в его кандидатской диссертации \cite{volkov}.


\section{Пространства многогранников и метрик}

\paragraph{Пространство многогранников.}
Обозначим через $\Phi$ пространство всех выпуклых многогранников в евклидовом пространстве, включая многогранники, вырождающиеся в плоский многоугольник.
Многогранники в $\Phi$ будут рассматриваться с точностью до движения пространства, а всё пространство $\Phi$ будет рассматриваться с естественной топологией (достаточно интуитивного понимания близости многогранников друг другу).

Через $\Phi_n$ будем обозначать многогранники в $\Phi$ с ровно $n$ вершинами.
Поскольку любой многогранник в $\Phi$ имеет по крайней мере три вершины,
пространство $\Phi$ разбивается на счётное число подмножеств $\Phi_3,\Phi_4,\dots$

\paragraph{Пространство многогранных метрик.}
Пространство всех многогранных метрик на сфере с суммой углов вокруг каждой точки не более $2\cdot\pi$ будет обозначаться $\Psi$.
При этом метрики в $\Psi$ будут рассматриваться с точностью до изометрии, а всё пространство будет рассматриваться с естественной топологией (опять же, достаточно интуитивного понимания близости двух метрик).

Точка на сфере, вокруг которой сумма углов строго меньше $2\cdot\pi$, будет называться \emph{существенной вершиной}.
Подмножество в $\Psi$, состоящее из метрик с ровно $n$ существенными вершинами, будет обозначаться $\Psi_n$.
Нетрудно видеть, что любая метрика в $\Psi$ имеет по крайней мере три существенные вершины,
так что $\Psi$ разбивается на счётное число подмножеств $\Psi_3,\Psi_4,\dots$

\paragraph{От многогранника к его поверхности.}
Напомним, что поверхность выпуклого многогранника является сферой с многогранной метрикой, такой, что сумма углов вокруг каждой точки не более $2\cdot\pi$.
Таким образом, переход от многогранника к его поверхности определяет отображение 
\[\iota\:\Phi\to \Psi.\]

Заметим также, что число вершин многогранника равно числу существенных вершин его поверхности.
Иначе говоря, $\iota(\Phi_n)\subset \Psi_n$ для любого $n\ge 3$.

\section{О доказательстве}

Используя обозначения, введённые в предыдущей секции, можно дать следующую более точную формулировку теоремы Александрова:

\begin{thm}{Переформулировка теоремы}
Для любого целого $n\ge 3$
отображение $\iota$ является биекцией из $\Phi_n$ в $\Psi_n$.
\end{thm}

Приведённый ниже набросок основан на оригинальном доказательстве А. Д. Александрова.
Главное в нём --- построение однопараметрического семейства многогранников, которое начинается с произвольного многогранника и заканчивается многогранником с поверхностью, изометричной данной.  
Подобный ход рассуждений называется \emph{методом непрерывности};
он часто используется в теории дифференциальных уравнений.

\medskip

Две части первой формулировки теоремы доказываются отдельно.

\parit{Часть \ref{thm:unique}.} Докажем, что отображение $\iota\:\Phi_n\to\Psi_n$ инъективно; иначе говоря, выпуклый многогранник определяется внутренней метрикой на своей поверхности с точностью до движения пространства.

Последнее утверждение является уточнением теоремы Коши о многогранниках. 
Его доказательство практически повторяет доказательство Коши.

Теорема Коши утверждает, что грани многогранника вместе с правилом склейки полностью определяют выпуклый многогранник;
её доказательство приводится во многих классических популярных текстах \cite{aigner-zigler,dolbilin,tabacnikov-fuks}.

\medskip

\parit{Часть \ref{thm:exist}.} Докажем, что отображение $\iota\:\Phi_n\to\Psi_n$ сюръективно.
Это часть доказательства разбивается на следующие леммы.

\begin{thm}{Лемма}
Пространство $\Psi_n$ связно для любого целого $n\ge 3$.
\end{thm}

Доказательство этой леммы несложное, но требует изобретательности; 
его можно провести явным построением непрерывного однопараметрического семейства  метрик в $\Psi_n$, соединяющего две данные метрики.
Такое семейство можно получить путём последовательного применения следующего построения и ему обратного.

Пусть $M$ --- сфера с метрикой из $\Psi_n$, а $v$ и $w$ --- две существенные вершины в $M$.
Разрежем $M$ вдоль кратчайшей линии от $v$ до $w$. 
Заметим, что кратчайшая не проходит через другие существенные вершины в $M$.
Далее заметим, что существует трёхпараметрическое семейство заплаток, которыми можно заклеить разрез так, чтобы полученная метрика осталась в $\Psi_n$; 
в частности, полученная метрика всё ещё будет иметь ровно $n$ существенных вершин (при этом вершины $v$ и $w$ могут перестать быть существенными).


\begin{thm}{Лемма}
Отображение $\iota\:\Phi_n\to\Psi_n$ открыто, 
то есть образ открытого множества в $\Phi_n$ открыт в $\Psi_n$.

В частности, для любого $n\ge 3$, множество $\iota(\Phi_n)$ открыто в~$\Psi_n$.
\end{thm}

Это утверждение очень близко к так называемой \emph{теореме об инвариантности области}, утверждающей, что непрерывное инъективное отображение между многообразиями равной размерности открыто.

Согласно части \ref{thm:exist}, отображение $\iota\:\Phi_n\to\Psi_n$ инъективно.
Доказательство теоремы об инвариантности области теоремы проходит в нашем случае, так как оба пространства $\Phi_n$ и $\Psi_n$ имеют размерность $3\cdot n-6$ и оба похожи на многообразия, хотя, формально говоря, таковыми не являются.
На более техническом языке можно сказать, что оба пространства $\Phi_n$ и $\Psi_n$ имеют естественную структуру $(3\cdot n-6)$-мерного \emph{орбиобразия} и отображение $\iota$ уважает эту структуру.

Мы только убедимся в том, что размерность обоих пространств $\Phi_n$ и $\Psi_n$ равна $3\cdot n-6$.

Выберем многогранник $P$ в $\Phi_n$.
Заметим, что $P$ однозначно определяется $3\cdot n$ координатами своих $n$ вершин.
При этом мы можем считать, что первая вершина совпадает с началом координат, у второй две координаты нулевые, а у третьей одна координата равна нулю; таким образом, для описания многогранников, близких к $P$, достаточно $(3\cdot n-6)$-параметров.
При этом, если многогранник не имеет симметрий, то это описание можно сделать взаимно-однозначным;
в этом случае окрестность точки $P$ в $\Phi_n$ является $(3\cdot n-6)$-мерным многообразием.
Если группа симметрий $P$ нетривиальна,
то данное описание не является взаимно-однозначным, однако это не влияет на размерность $\Phi_n$.

Случай многогранных метрик аналогичен.
Необходимо построить разбиение сферы на плоские треугольники используя только существенные вершины.
По формуле Эйлера, получаем, что число рёбер в данном разбиении равно $3\cdot n-6$.
При этом длины рёбер однозначно определяют метрику и небольшое изменение длин рёбер даёт метрику с теми же свойствами.


\begin{thm}{Лемма}
Отображение $\iota\:\Phi_n\to\Psi_n$ замкнуто, 
то есть образ замкнутого множества в $\Phi_n$ замкнут в $\Psi_n$.

В частности, для любого $n\ge 3$ множество $\iota(\Phi_n)$ замкнуто в~$\Psi_n$.
\end{thm}

Выберем замкнутое множество $Z$ в $\Phi_n$.
Обозначим через $\bar Z$ замыкание множества $Z$ в $\Phi$; заметим, что $Z=\Phi_n\cap \bar Z$.
Предположим, что $P_1,P_2,\dots\in Z$ --- последовательность многогранников, сходящихся к многограннику $P_\infty\in\bar Z$.
Заметим, что $\iota(P_n)$ сходится к $\iota(P_\infty)$  в $\Psi$.
В частности, $\iota(\bar Z)$ замкнуто в $\Psi$.

Поскольку $\iota(\Phi_n)\subset \Psi_n$ для любого $n\ge 3$ получаем, что  $\iota (Z)=\iota(\bar Z)\cap \Psi_n$, чем и завершается доказательство леммы. 

\medskip

Итак, множество $\iota(\Phi_n)$ --- непусто, замкнуто и открыто в $\Psi_n$, при этом $\Psi_n$ связно для любого $n\ge 3$.
Значит, $\iota(\Phi_n)=\Psi_n$; то есть, $\iota\:\Phi_n\z\to\Psi_n$ сюръективно.
\qeds

\parbf{Благодарности.} Мы благодарим С. Б. Александер, Ю. Д. Бураго и Ж. Цукахара за помощь. 
%Статья написана при частичной поддержке РФФИ грант 20-01-00070 и ННФ грант DMS-2005279.

\sloppy
\printbibliography[heading=bibintoc]
\fussy

\end{document}
