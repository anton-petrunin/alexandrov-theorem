\documentclass[oneside,a4paper]{article}
\usepackage{en-note}
\usepackage{indentfirst}
\hypersetup{
pdftitle={Alexandrov's embedding theorem},
pdfauthor={Nina Lebedeva and Anton Petrunin}
}


\begin{document}
%\pagestyle{empty}\renewcommand\includegraphics[2][{}]{}

\title{Alexandrov's embedding theorem}
\author{Nina Lebedeva and Anton Petrunin}
\date{}
\maketitle

\begin{abstract}
Alexandrov's embedding theorem provides a complete description of the intrinsic geometry of surfaces of convex polyhedrons.
We give a sketch of its proof.
\end{abstract}

\section{Introduction}

Intrinsic distance between two points on the surface of a convex polyhedron is defined as the length of a shortest curve on the surface between these points.

Recall that the sum of angles at the tip of a convex polyhedral angle is less than $2\cdot\pi$;
this statement can be found in a school textbook \cite[§~48]{kiselyov-3D}.

It is easy to see that the surface of a convex polyhedron is homeomorphic to the sphere.
Therefore the statements above imply that the surface of a convex polyhedron equipped with its intrinsic metric is an example of a \emph{polyhedral metric on the sphere with the sum of angles around each vertex at most $2\cdot\pi$};
a metric is called \emph{polyhedral} if the sphere admits a triangulation such that every triangle is congruent to a plane triangle.

Alexandrov's theorem states that the converse holds if one includes in the consideration \emph{twice covered polygons}.
In other words, we assume that a polyhedron can degenerate to a plane polygon;
in this case, its surface is defined as two copies of the polygon glued along their boundary.

Further, we assume that a polyhedron can degenerate to a plane polygon.

\begin{thm}{Alexandrov's theorem}
\begin{enumerate}[I.]
\item\label{thm:exist}
A polyhedral metric on the sphere is isometric to the surface of a convex polyhedron if and only if the sum of angles around each of its vertex is not greater than $2\cdot\pi$.

\item\label{thm:unique} 
Moreover, a convex polyhedron is defined up to congruence by the intrinsic metric on its surface.
\end{enumerate}

\end{thm}

A. D. Alexandrov has many remarkable theorems, but in our opinion, this theorem is the most remarkable.
At the same time, its proof is elementary;
it could be explained to anyone familiar with basic topology.

This theorem has many applications.
In particular, it is used in the proof of its generalization \cite{alexandrov-1948} that gives a complete description of intrinsic metrics on the sphere that are isometric to convex surfaces in the Euclidean space.
The latter statement is fundamental in a branch of modern mathematics --- the so-called \emph{Alexandrov geometry}.

The first part is central; it is called the \emph{existence theorem}.
The second part is called the \emph{uniqueness theorem}; it is a slight variation of Cauchy's theorem about polyhedrons.
(There is another uniqueness theorem of Alexandrov that generalizes Minkowski's theorem about  polyhedrons.)

\begin{wrapfigure}{r}{30mm}
\vskip-0mm
\centering
\includegraphics{mppics/pic-10}
\vskip-0mm
\end{wrapfigure}

According to the theorem, a convex polyhedron is completely defined by the intrinsic metric of its surface.
In particular, knowing the metric we could find the position of the edges.
However, in practice, it is not easy to do.
For example, the surface glued from a rectangle as shown on the diagram defines a tetrahedron.
Some of the glued lines appear inside facets of the tetrahedron and some edges (dashed lines) do not follow the sides of the rectangle.

The theorem was proved by A. D. Alexandrov in 1941 \cite{alexandrov-1941};
we will present a sketch of his proof.
A complete proof is nicely written by A. D. Alexandrov in his book~\cite{alexandrov}.
Yet another proof was found by Yu.~A.~Volkov in his thesis \cite{volkov};
it uses a deformation of three-dimensional polyhedral space.


\section{Space of polyhedrons and metrics}

\paragraph{Space of polyhedrons.}
Let us denote by $\Phi$ the space of all convex polyhedrons in the Euclidean space,
including polyhedrons that degenerate to a plane polygon.
Polyhedra in $\Phi$ will be considered up to a motion of the space, 
and the whole space $\Phi$ will be considered with the natural topology (an intuitive meaning of closeness of two polyhedrons should be sufficient).  

Further, denote by $\Phi_n$ the polyhedrons in $\Phi$ with exactly $n$ vertices.
Since any polyhedron has at least 3 vertices, the space $\Phi$ admits a subdivision into a countable number of subsets $\Phi_3,\Phi_4,\dots$

\paragraph{Space of polyhedral metrics.}
The space of polyhedral metrics on the sphere with the sum of angles around each point at most $2\cdot\pi$ will be denoted by $\Psi$.
The metrics in $\Psi$ will be considered up to an isometry, and the whole space $\Psi$ will be equipped with the natural topology (again, an intuitive meaning of closeness of two metrics is sufficient).

A point on the sphere with the sum of angles strictly less than $2\cdot\pi$ will be called an \emph{essential vertex}.
The subset of $\Psi$ of all metrics with exactly $n$ essential vertices will be denoted by $\Psi_n$.
It is easy to see that any metric in $\Psi$ has at least 3 essential vertices.
Therefore $\Psi$ is subdivided into countably many subsets
 $\Psi_3,\Psi_4,\dots$

\paragraph{From a polyhedron to its surface.}

Recall that the surface of a convex polyhedron is a sphere with a polyhedral metric such that the sum of angles around each point is at most $2\cdot\pi$.
Therefore passing from a polyhedron to its surface defines a map
\[\iota\:\Phi\to \Psi.\]

Note that the number of vertices of a polyhedron is equal to the number of essential vertices of its surface.
In other words, $\iota(\Phi_n)\subset \Psi_n$ for any $n\ge 3$.

\section{About the proof}

Using the notation introduced in the previous section, we can give the following more exact formulation of Alexandrov's theorem: 

\begin{thm}{Reformulation}
For any integer $n\ge 3$,
the map $\iota$ is a bijection from $\Phi_n$ to~$\Psi_n$.
\end{thm}

We sketch the original proof of A. D. Alexandrov.
It is based on the  construction of a one-parameter family of polyhedrons that starts at arbitrary polyhedron
and ends at an arbitrary polyhedron with its surface isometric to the given one.
This type of argument is called the \emph{continuity method}; it is often used in the theory of differential equations.

\medskip

The two parts of the first formulation will be proved separately.

\parit{Part \ref{thm:unique}.} Let us show that the map $\iota\:\Phi_n\to\Psi_n$ is injective;
in other words, a convex polyhedron is defined by the intrinsic metric on its surface up to a motion of the space.

The last statement is analogous to the Cauchy theorem about polyhedrons,
and the proof goes along the same lines. 

The Cauchy theorem states that facets of a polyhedron together with the gluing rule completely describe a convex polyhedron;
its proof is given in many classical popular texts \cite{aigner-zigler,dolbilin,tabacnikov-fuks}.

\medskip

\parit{Part \ref{thm:exist}.}
Let us prove that $\iota\:\Phi_n\to\Psi_n$ is surjective.
This part of the proof is subdivided into the following lemmas:

\begin{thm}{Lemma}
For any integer $n\ge 3$, the space $\Psi_n$ is connected.
\end{thm}

The proof of this lemma is not complicated, but it requires ingenuity;
it can be done by the direct construction of a one-parameter family of metrics in $\Psi_n$ that connects two given metrics.
Such a family can be obtained by а sequential application of the following construction and its inverse.

Let $M$ be a sphere with metric from $\Psi_n$.
Suppose $v$ and $w$ are essential vertices in $M$.
Let us cut $M$ along a shortest line from $v$ to $w$.
Note that the shortest line cannot pass thru an essential vertex of $M$.
Further, note that there is a three-parameter family of patches that can be used to patch the cut so that the obtained metric remains in $\Psi_n$;
in particular, the obtained metric has exactly $n$ essential vertices (after the patching, the vertices $v$ and $w$ may become inessential).


\begin{thm}{Lemma}
The map $\iota\:\Phi_n\to\Psi_n$ is open, 
that is, it maps any open set in $\Phi_n$ to an open set in $\Psi_n$.

In particular, for any $n\ge 3$, the image $\iota(\Phi_n)$ is open in~$\Psi_n$.
\end{thm}

This statement is very close to the so-called \emph{invariance of domain theorem};
the latter states that a continuous injective map between manifolds of the same dimension is open.

According to part \ref{thm:unique}, $\iota$ is injective.
The proof of the invariance of domain theorem can be adapted to our case since both spaces $\Phi_n$ and $\Psi_n$ are $(3\cdot n-6)$-dimensional and both look like manifolds, altho, formally speaking, they are \emph{not} manifolds.
In a more technical language, $\Phi_n$ and $\Psi_n$ have the natural structure of $(3\cdot n-6)$-dimensional \emph{orbifolds},
and the map $\iota$ respects the \emph{orbifold structure}.

We will only show that both spaces $\Phi_n$ and $\Psi_n$ are $(3\cdot n-6)$-dimensional.

Choose a polyhedron $P$ in $\Phi_n$.
Note that $P$ is uniquely determined by the $3\cdot n$ coordinates of its $n$ vertices.
We can assume that the first vertex is the origin, the second has two vanishing coordinates and the third has one vanishing coordinate; therefore, all polyhedrons in $\Phi_n$ that lie sufficiently close to $P$ can be described by $3\cdot n-6$ parameters.
If $P$ has no symmetries then this description can be made one-to-one;
in this case, a neighborhood of $P$ in $\Phi_n$ is a $(3\cdot n-6)$-dimensional manifold.
If $P$ has a nontrivial symmetry group, then this description is not one-to-one but it does not have an impact on the dimension of $\Phi_n$.

The case of polyhedral metrics is analogous.
We need to construct a subdivision of the sphere into plane triangles using only essential vertices.
By Euler's formula, there are exactly $3\cdot n-6$ edges in this subdivision.
Note that the lengths of edges completely describe the metric, and slight changes of these lengths produce a metric with the same property.

\begin{thm}{Lemma}
The map $\iota\:\Phi_n\to\Psi_n$ is closed;
that is, the image of a closed set in $\Phi_n$ is closed in $\Psi_n$.

In particular, for any $n\ge 3$, the set $\iota(\Phi_n)$ is closed in~$\Psi_n$.
\end{thm}

Choose a closed set $Z$ in $\Phi_n$.
Denote by $\bar Z$ the closure of $Z$ in $\Phi$; note that $Z=\Phi_n\cap \bar Z$.
Assume $P_1,P_2,\dots\in Z$ is a sequence of polyhedrons that converges to a polyhedron $P_\infty\in\bar Z$.
Note that $\iota(P_n)$ converges to $\iota(P_\infty)$ in $\Psi$.
In particular, $\iota(\bar Z)$ is closed in $\Psi$.

Since $\iota(\Phi_n)\subset \Psi_n$ for any $n\ge 3$, we have  $\iota (Z)=\iota(\bar Z)\cap \Psi_n$;
that is, $\iota (Z)$ is closed in $\Psi_n$. 

\medskip

Summarizing, $\iota(\Phi_n)$ is a nonempty closed and open set in $\Psi_n$, and $\Psi_n$ is connected for any $n\ge 3$.
Therefore, $\iota(\Phi_n)=\Psi_n$; that is, $\iota\:\Phi_n\z\to\Psi_n$ is surjective.
\qeds

\parbf{Acknowledgments.} We want to thank S. Alexander, Y. Burago, and J. Tsukahara for help. 
%The authors were partially supported by RFBR grant 20-01-00070 and NSF grant DMS-2005279.

\sloppy
\printbibliography[heading=bibintoc]
\fussy

\end{document}
