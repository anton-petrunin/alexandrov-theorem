 \documentclass[oneside,a4paper]{article}
\usepackage{ru-note}
\usepackage{indentfirst}
\hypersetup{
pdftitle={Теорема Александрова о вложении многогранников},
pdfauthor={Нина Лебедева и Антон Петрунин}
}
%\renewcommand*{\HyperDestNameFilter}[1]{\jobname-#1}

\begin{document}

\title{Теорема Александрова о\\ вложении многогранников}
\author{Нина Лебедева и Антон Петрунин}
\date{}
\maketitle

\begin{abstract}
Мы приводим набросок доказательства теоремы Александрова о том, что любая многогранная метрика на сфере с неотрицательной кривизной изометрична поверхности некоторого выпуклого многогранника.
\end{abstract}

\section{Введение}

Нас будет интересовать \emph{внутренняя метрика} на поверхности выпуклого многогранника;
то есть длина кратчайшей кривой на поверхности соединяющей две данные точки.
Напомним, что сумма углов при вершине многогранного выпуклого угла меньше $2\cdot \pi$; это утверждение можно найти в учебнике А. П. Киселёва \cite[§~325]{kiselyov}.

Нетрудно видеть, что поверхность выпуклого многогранника гомеоморфна сфере.
Значит поверхность выпуклого многогранника наделённая естественной внутренней метрикой
является примером \emph{многогранной метрики на двумерной сфере с неотрицательной кривизной}; то есть,
она допускает триангуляцию, 
такую что каждый её треугольник равен плоскому треугольнику, а сумма углов вокруг каждой вершины не превосходит $2\cdot\pi$.

Теорема Александрова гласит, что обратное верно если включить в рассмотрение многогранники вырождающиеся в плоские многоугольники.
В этом случае под его поверхностью понимается две копии многоугольника склеенные по границе
(можно думать что у многоугольника есть две стороны и чтобы перейти на другую сторону следует обогнуть периметр).

Везде далее мы предполагаем, что многогранник может вырождаться в плоский многоугольник.


\begin{thm}{Теорема Александрова}
Многогранная метрика на сфере изометрична поверхности выпуклого многогранника тогда и только тогда, когда сумма углов при любой её вершине не превосходит $2\cdot\pi$.
Более того, многогранник определяется метрикой на своей поверхности с точностью до конгруэнтности.
\end{thm}

Эта теорема послужила отправной точкой в существенной перестройке теории выпуклых поверхностей и несомненно одна из самых удивительных теорем доказанных А. Д. Александровым.
При этом её доказательство вполне элементарно, так что может быть объяснено первокурснику матмеха.
Мы приведём набросок доказательства; полное доказательство можно найти в замечательно написанной книжке Александра Даниловича~\cite{alexandrov}.


\section{Пространства многогранников и метрик}

\paragraph{Пространство многогранников.}
Обозначим через $\Phi$ пространство всех выпуклых многогранников в евклидовом проестранстве, включая многогранники вырождающиеся в плоский многоугольник.
Многогранники в $\Phi$ будут рассматриваться с точностью до движения пространства сохраняющего ориентацию, а всё пространство $\Phi$ будет рассмтариватся с естественной топологией (достаточно интуитивного понимания того, что два многогранника близки).

Через $\Phi_n$ будем обозначать многогранники в $\Phi$ с ровно $n$ вершинами.
Поскольку любой многогранник в $\Phi$ имеет по крайней мере 3 вершины;
$\Phi$ разбивается в на счётное число подмножеств $\Phi_3,\Phi_4,\dots$

\paragraph{Пространство многограных метрик.}
Пространство всех многогранных метрик на сфере с неотрицательной кривизной будет обозначаться $\Psi$.
При этом метрики в $\Psi$ будут рассматриваться с точностью до изометрии сохраняющей ориентацию сферы, а всё пространство будет рассмтариватся с естественной топологией (опять же, достаточно интуитивного понимания того, что две метрики близки).

Точка вокруг которой сумма углов строго меньше $2\cdot\pi$ будет называться \emph{существенной вершиной}.
Подмножество $\Psi$ состоящее из метрик с ровно $n$ существенными вершинами будет обозначаться $\Psi_n$.
Нетрудно видеть, что любая многогранная метрика в $\Psi$ имеет по крайней мере 3 существенных вершины;
и значит $\Psi$ разбивается в на счётное число подмножеств $\Psi_3,\Psi_4,\dots$.

\paragraph{От многогранника к его поверхности.}
Напомним, что поверхность выпуклого многогранника является многогранной метрикой на сфере с неотрицательной кривизной.
Заметим, что ориентация евкидова пространства определяет ориентацию поверности многогранника.
Таким образом переход от многогранника к его поверхности определяет отображение 
\[\iota\:\Phi\to \Psi.\]

Заметим также, что число вершин многогранника равно числу существенных вершин его поверхности;
то есть вершин с суммарный угол вокруг которых строго меньше $2\cdot\pi$.
Иначе говоря, $\iota(\Phi_n)\subset \Psi_n$ для любого $n\ge 3$.

\section{О доказательстве}

Используя обозначения, введённые в предыдущей секции, можно дать следующую более точную формулировку теоремы Александрова:

\begin{thm}{Переформулировка}
Для любого $n\ge 3$,
отображение $\iota$ является биекцией из $\Phi_n$ в $\Psi_n$.
\end{thm}


\parit{Набросок доказательства.}
Во первых нам надо доказать, что отображение $\iota\:\Phi_n\to\Psi_n$ инъетивно; иначе говоря многогранник определяется внутренней метрикой на своей поверхности с точностью до движения пространства сохраняющего ориентацию.

Последнее утверждение является уточнением теоремы Коши о многогранниках и его доказательство практически повторяет доказательство Коши.

Теорема Коши утверждает, что грани многогранника вместе с правилом склейки полностью определяют выпуклый многогранник;
её доказательство содержится например в отлично написанной популярной книжке С. Л. Табачникова и Д. Б. Фукса \cite{tabacnikov-fuks}.

Остаётся доказать, что $\iota\:\Phi_n\to\Psi_n$ сюрьетивно.
Эта часть доказательства разбивается следующие несложные леммы:

\begin{thm}{Лемма}
На пространства $\Phi_n$ и $\Psi_n$ связанны.
\end{thm}

Доказательство этого предложения требует изобретательности.
Но его можно провести явным построением неопрерывного однопараметрического семейства многогранников в $\Phi_n$ (или метрик в $\Psi_n$) сочиняющее два данных многогранника (соответственно метрики). 

\begin{thm}{Лемма}
На пространствах $\Phi_n$ и $\Psi_n$ можно ввести естественную структуру $(3\cdot n-6)$-мерного орбиобразия без границы; то есть окрестность любой точки в этих пространствах допускает естественную параметризацию фактором $\mathbb{R}^{3\cdot n-6}/\Gamma$, где $\Gamma$ --- конечная группа поворотов $\mathbb{R}^{3\cdot n-6}$ без отражений.
\end{thm}

Действительно, многогранник $P$ в $\Phi_n$ однозначно определяется $3\cdot n$ координатами своих $n$ вершин.
При этом мы можем считать, первая вершина совпадает с началом координат, вторая лежит на оси $x$, а третья в плоскости $(x,y)$; таким образом, для описания многогранников близких $P$ достаточно $3\cdot n-6$ координат.
При этом, если многогранник не имеет симметрий, это описание взаимно однозначно.
При наличии у $P$ группы симметрий $\Gamma$ сохраняющей ориентацию, следует по ней профакторизовать. 

Случай многогранных метрик аналогичен.
Необходимо построить разбиение поверхности на плоские треугольники используя только сущёетвенные вершины.
По формуле Эйлера, получаем, что число рёбер в данном разбиении равно $3\cdot n-6$ и небольшое изменение длин рёбер даёт метрику с теми же свойствами.
Далее профакторизовать по симметриям исходной метрики.

Для ориентированных орбиобразий выполняется так называемая \emph{теорема об инвариантности области} --- она влечёт, что $\iota$ переводит открытые множестве из $\Phi_n$ в открытые множества $\Psi_n$.
В частности, мы получаем следующее:


\begin{thm}{Лемма}
Для любого $n\ge 3$, множество $\iota(\Phi_n)$ открыто в~$\Psi_n$.
\end{thm}

Далее, предположим $P_1,P_2,\dots\in \Phi_n$ --- последовательность многогранников, сходящихся к многограннику $P_\infty$.
Заметим, что $\iota(P_n)$ сходится к $\iota(P_\infty)$.
В частности, если $\bar \Phi_n$ обозначет замыкание $\Phi_n$ в $\Phi$,
то $\iota(\bar\Phi_n)$ замкнуто в $\Psi_n$.
Отсюда получаем следующее:

\begin{thm}{Лемма}
Для любого $n\ge 3$, множество $\iota(\Phi_n)$ замкнуто в $\Psi_n$.
\end{thm}


И так, $\iota(\Phi_n)$ --- замкнуто и открыто подмножество $\Psi_n$, при этом оба множества $\Phi_n$ и $\Psi_n$ связаны.
Значит образ $\iota(\Phi_n)$ покрывает $\Psi_n$; то есть, $\iota\:\Phi_n\to\Psi_n$ сюрьетивно.
\qeds

  

\sloppy
\printbibliography[heading=bibintoc]
\fussy

\end{document}
